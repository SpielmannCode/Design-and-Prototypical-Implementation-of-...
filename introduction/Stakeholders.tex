% authors: Nicolas Spielmann & Michael Bucher
\section{Stakeholders}\label{chapter:introduction:Stakeholder}
In the following, the main interested parties of a ticketing platform are characterized and their intentions and needs are elaborated.

\subsection{Guests}
Guests are persons, who want to attend an event. They want to be able to buy tickets for a fair price and resell their owned tickets in case they can not attend an event. Furthermore, they want to have fair odds when trying to obtain a ticket and be certain, that the ticket they bought is not counterfeit. In the current landscape, the initial sale of tickets is not highly regulated, therefore, scalpers exist, that make it their business to buy tickets in bunch and resell them on the aftermarket for a higher price, making it hard for real guests to obtain tickets when there is a high demand. Furthermore, the validity of the tickets sold on the aftermarket can often not be guaranteed.

\subsection{Event hosts}\label{chapter:introduction:Stakeholder:EventHost}
Event hosts are entities, that organize events. They are responsible for making deals with various artists and venues to organize and carry out events. Therefore, they want to be able to create events, issue tickets and sell them for a defined price to event guests. In the current landscape, they experience two main issues. Usually, ticket organizations claim a large cut from the actual ticket price, so that the event organizers and the content creators are left with much less. Hence, a substitute solution for ticket organizations which leads to less deductions are sought. Further, potential ticket scalpers in high demand events do not only lead to overpriced tickets for guests, but can also affect the event venues. Considering a scenario where a ticket scalper is able to buy 100 tickets for CHF 50 each and resell them at a price of CHF 150, he can already make profit with only selling 34 out of those 100. In that case however, only those 34 guests instead of 100 will appear at the venue and consume food, drinks and other offers. Regarding this factor, the event venue has an incentive to eliminate potential ticket scalpers.

\subsection{Content Creators}
Content Creators are entities, that produce content such as music, video or comedy shows, and want to present this content live to an audience. For this, they contact the event host and venues to plan an event. Their main goal is to earn money by performing. However, they are also keen on performing for a maximum amount of guest. In the current landscape, as described in the Section \ref{chapter:introduction:Stakeholder:EventHost}, there might be the case, that a scalper may keep some tickets, as he already earned money. This may lead to the content creators having to perform for a non fully filled venue, even if the venue was sold out. This might reduce the energy of the performance and demotivate the content creators.
