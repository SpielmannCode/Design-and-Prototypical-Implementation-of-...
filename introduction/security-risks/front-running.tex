% author: Simon Bachmann

\subsection{Front Running}\label{subsection:front-running}

Every transaction on a public BC is visible to everybody in the network. However, transactions are only considered finalized when they are accepted by a miner and included in a block. The time of publishing a transaction and time of finalization depends on the amount of transaction fees a user is willing to pay. If the network is clogged with unfinalized transactions, it is possible to get a transaction validated with higher transaction fees because miners want to maximize profit and generally include the transactions that offer the highest fees. This leads to the problem that one transaction can be finalized before some other transaction that was published beforehand. 

An initial coin offering (ICO) is a funding mechanism where investors send tokens to a SC. These funds can then be used by the project team to develop the promised platform or service. In return the investors receive a platform specific token that can be used to consume the service. For some ICOs the maximal amount of funding is limited. If that is the case, investors pay high gas fees to have their transaction confirmed quickly. In the past, this has led to a lot of frustration of investors because transaction fees have spiked disproportionately high compared to the average transaction fees when no popular ICO is taking place \cite{bat-ico-tx-fees} \cite{ico-paradox}.

Since event tickets are in many aspects similar to ICOs tokens such as their limited availability, the same problem occurs. When the demand for tickets exceeds its supply, buyers will try to pay higher transaction fees in order to obtain a ticket before someone else. This might also result in disproportionately high transaction fees. 
