% author: Simon Bachmann

\subsection{Front Running}\label{subsection:front-running}

Every transaction on a public blockchain is visible to everybody in the network. However, transactions are only considered finalized when they are accepted by a miner and included in a block. The time of publishing a transaction and time of finalization depends on the amount of transaction fees a user is willing to pay. If the network is clogged with unfinalized transactions, it is possible to get a transaction validated with higher transaction fees because miners want to maximize profit and generally include the transactions that offer the highest fees. This leads to the problem that I transaction can be finalized before some other transaction that was published beforehand. 

During an ICO, the issue occurs where a limited amount of tokens are available. In the past, this has led to a lot of frustration of investors because transaction fees have spiked disproportionately \cite{bat-ico-tx-fees} \cite{ico-paradox}.

Since event tickets are in many aspects similar to ICO tokens, the same problem occurs. When the demand for tickets exceeds its supply, buyers will try to pay higher transaction fees in order to obtain a ticket before someone else. This might also result in unproportionally high transaction fees. 
