\subsection{Lottery Manipulation}\label{subsection:lottery-manipulation}
In order to make a fair pre-sale of event tickets, every user wanting a ticket for a certain event should have the same probability to receive this ticket. One way of addressing this need is to issue a pre-sale, where everybody can register their intention to buy a ticket. Instead of first come, first served, it would then be possible to have a lottery, where everybody gets a ticket with the probability defined by the number of tickets available divided by the ticket wanted.

However, this trivial approach only works, when every user is only buying one ticket. Most of the time, it is desired by the user to be able to buy tickets for a certain group (Friends, Family, Coworkers etc.). When this feature should be implemented, it is no longer trivial to calculate the probability a buyer should get his \textit{n} tickets. 

The probability of being able to buy \textit{n} tickets has to be calculated in a way, that the lottery still is fair. it it crucial to not be able to maximise the probability of getting a ticket by grouping up and create multiple request of buying \textit{n} tickets.

