% author: Simon Bachmann

\subsection{Location-Based Addressing for Metadata Storage}\label{subsection:metadata-storage}

Storing data on a public BC is expensive since every node in the network creates a copy of that data on its machine. Thus, many decentralized applications only store the hashes of files on the BC while storing the actual metadata somewhere else. With this design, it is possible to check whether the actual file was modified because any change in the file would result in a different hash. However, if the actual file is lost, the data cannot be reconstructed from the hash and is permanently lost. Thus, redundancy and availability play important roles when designing a decentralized application.

It is desired to only store the information on-chain which is relevant to the functionality of the SC. Metadata such as event details that are not relevant for any of the methods in the SC must no be stored on-chain.

However, storing a metadata file on a server is not an ideal solution since the location of the server can change or even no longer exist. Even if the file was replicated on a different machine, the nodes in the network do not know where the new location is, if the SC does not keep a list of all the hosts. Thus, location-based addressing for metadata is therefore prone to censorship and loss of data. 
