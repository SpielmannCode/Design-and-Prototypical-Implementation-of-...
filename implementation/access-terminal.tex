% Author: Michael Bucher
\section{Access Terminal}\label{imp:terminal}
The access terminal is a web application that supports the connection to the access control backend to handle ticket validation at the venue. Its implementation is structured into the two components Connection and Terminal.

\subsection{Connection}
The Connection component handles the initial connection to the backend. It contains a simple form to define the necessary parameters for registration in the backend. When the registration API call was successful, the application stores the returned terminal id and routes to the Terminal view.

\subsection{Terminal}
The Terminal component first requests a new \textit{randId} on line 2 in Figure \ref{code:terminal-start}. This message is then displayed in a QR code together with the terminal id to be scanned by the guest-client application. Then an interval is started, which executes the method \textit{handleStatus} (Figure \ref{code:terminal-status-handling}) continuously. This method requests the terminal's status in the backend and executes the according methods depending on the return. The method \textit{showQR} on line 5 simply displays the current values for the \textit{randId} and terminal id as a QR code. On the lines 7 and 9 are the states \textit{granted} and \textit{denied} handled. E.g. when the status changes to \textit{granted}, the method \textit{handleGrantedStatus} as in Figure \ref{code:terminal-handle-granted} is executed. It fetches the amount of tickets that were validated and displays this number. When the guest has clicked on the terminal and entered the location, a new \textit{randId} is requested, the status in the backend changes back to \textit{pending}, the newly retrieved \textit{randId} is displayed in the QR code and the terminal is ready to receive the next updated status from the backend.

\begin{figure}[H]
    \lstinputlisting[language=Java]{code-snippets/terminalCreated.js}
    \caption{Terminal creation}
    \label{code:terminal-start}
\end{figure}

\begin{figure}[H]
    \lstinputlisting[language=Java]{code-snippets/handleStatus.js}
    \caption{Terminal - General status handling}
    \label{code:terminal-status-handling}
\end{figure}

\begin{figure}[H]
    \lstinputlisting[language=Java]{code-snippets/terminalGrantedStatus.js}
    \caption{Terminal - Granted status handling}
    \label{code:terminal-handle-granted}
\end{figure}
