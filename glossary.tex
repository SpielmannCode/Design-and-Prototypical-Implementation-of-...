\chapter*{Glossary}
\addcontentsline{toc}{chapter}{Glossary}
\markboth{GLOSSARY}{}

\begin{description}
  \item[Base64] A binary-to-text algorithm that allows to convert data from various formats to plain text.
  \item[Ethereum Improvement Proposal] A proposal for an extension, change or potential improvement to the Ethereum platform, including the core protocol specifications, client APIs and the contract standards. Accepted Ethereum Improvement Proposals become standards on the Ethereum platform.
  \item[Fungible tickets] Tickets that are indifferent to other tickets of the same type such as a standing room ticket.
  \item[InterPlanetary File System] A protocol and peer-to-peer network to store data in a distributed file system.
  \item[Know Your Customer] Guidelines in financial services, that require the business to make an effort to verify the identity, suitability and risk involved when maintaining a business relationship. The main purpose is to prevent to be used for criminal elements.
  \item[Metamask] A browser plugin for interacting with the Ethereum BCs.
  \item[Mnemonic seed phrase] An ordered list of words which contain the information needed to access or recover a BC wallet.
  \item[Non-fungible tickets] Tickets that are uniquely identified with one-of-a-kind properties such as a seat number.
  \item[Optical Character Recognition] An electronic conversion of images of text (typed, handwritten or printed) into machine-encoded text.
  \item[Primary Market] The original marketplace where event hosts directly sell tickets to their event to event guests.
  \item[Quick response code] A type of matrix barcode.
  \item[Unix time stamp] A form of saving a specific time. It is the number of seconds that passed since January 1st, 1970 UTC and can be used as a time format to save a date as a number.
\end{description}
