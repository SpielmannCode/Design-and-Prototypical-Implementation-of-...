\chapter{Summary}\label{chapter:summary}

The open architecture of the proposed ticketing system benefits all stakeholders in the system. Event guests profit from fair ticketing pricing even on the aftermarket. Tickets cannot be sold for higher prices than the original cost of a ticket. Ticket scalpers have no scope of action and are prohibited by the protocol to benefit from the inefficiencies of today's ticketing market.

Since the data on Ethereum is publicly available, anyone can build an interface and become a GUI provider. GUI providers are incentivized to build appealing frontend applications in order to attract more users. They are compensated with every ticket sale that is initiated from their platform. A fair market of attractive and user-friendly frontend applications that connect to the same data can emerge.

With this open design, event guests can connect to their favourite ticketing GUI provider and they do not need to check whether a ticket for the same event is offered on any other platform since they all connect to the same data. The queuing architecture enables buyers and sellers to transact with each other. Also, there is no trust needed when exchanging a ticket on the aftermarket. The seller cannot fool the buyer by not sending the ticket after receiving the money. The SC guarantees that the ownership is only transferred if the money is sent.

Furthermore, the proposed architecture creates a new market for identity approvers which are needed in a decentralized system since there are no restrictions from creating a wallet and interacting with the platform. Identity approvers are incentivized to act trustworthy since they are financially compensated if they are chosen by the event hosts. Their approval on the BC can even be integrated into other services on the Ethereum BC. 

The social trust certificates creates a layer of trust that is higher than current online ticket platforms without the need of a trusted third party. Aggregating ownership proofs across multiple social profiles in the event listing increases the legitimacy of an event. 

The proposed presale design guarantees a fair distribution if the demand for an event is higher than its supply. Furthermore, it prevents front running attacks and should not lead to sudden surges in gas prices.

An event host can specify which currency is used for buying tickets. Specifically, ETH and any ERC20 token can be used as a mean of payment. There are many different implementations of stable coins in the Ethereum ecosystem. By allowing any ERC20 token to be specified, the host is not exposed to the price fluctuations of the volatility of ETH. 

With the host client web application hosts can easily create and manage an event and retrieve useful information about its state.

The guest client web application allows users to buy tickets once the initial hurdle of setting up an Ethereum wallet for a non technical user is overcome. The integrated and tightly coupled aftermarket provides a novelty function that is not available on most ticketing platforms as of today. Through this both the question of validity when buying a ticket from another user, as well as the inflated prices on tickets with high demand are addressed. The guest client application provides a user-friendly interface to all building blocks of the proposed SC-based ticketing platform.

As one main goal of this implementation was to prohibit scalpers from misusing the the secondary market of the ticket industry for their own profit. To achieve this, the design proposes to identify the guest and only grant the to buy a limited amount of tickets. This crucial part of identifying and approving the identity of a guest and linking the identity to an Ethereum Address is done by the identity approver application. As the proposed implementation is just an example and the Identity SC is used to store the identities, every Entity can act as an identity approver.

The access control application is the last component in the proposed solution, that is required to provide a solution for the full life cycle of a ticket. All the other components are mainly used to acquire tickets. The access control application is used to check the ownership of the required ticket. It guarantees, that only guest holding a valid ticket can enter the event. 
