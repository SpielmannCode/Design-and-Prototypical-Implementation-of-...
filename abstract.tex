\chapter*{Abstract}
\addcontentsline{toc}{chapter}{Abstract}

\selectlanguage{english}
% author: Simon Bachmann

The ticketing industry is and has long been subject to counterfeiting and profiteering. The secondary market for high demand events attracts ticket scalpers, that hoard tickets and try to sell them at prices well above the initial value. This can be seen as a normal market force, that allows quick movers to make a profit. However, it is a concern for many artists and organizers who lose control over their ticket pricing and distribution. For example, Ed Sheeran's Benefit Concert at Albert Hall (UK) in 2017 for the Teenage Cancer Trust was abused for heavy profiteering where tickets with a face value of 75 pounds have been resold for up to 2330 pounds \cite{ed-sheeran-concert-ticket-prices}. Beyond the reputation problem and fairness concern, the secondary market is also problematic due to fraud. Tickets are often fake or have already been used rendering them invalid. 

The blockchain (BC) technology offers an opportunity to remedy these issues. In this project an open and decentralized platform for distributing event tickets and regulating the aftermarket is designed, implemented and evaluated. The system operates through smart contracts (SC) on the Ethereum BC, enabling users to verify the validity of tickets for a given event, ending counterfeit tickets and allowing the secure resale of tickets based on the organizers pricing to avoid ticket scalping. Furthermore, a mechanism to help users detect fraudulent events is discussed. Also, a system that invalidates tickets in a fast and efficient manner is designed and implemented.

\newpage
\selectlanguage{german}
% author: Michael Bucher

Die Ticketindustrie muss sich immer wieder mit Ticketf{\"a}lschungen und Gesch{\"a}ftemachereien auseinandersetzen. Der Sekund{\"a}rmarkt f{\"u}r Events mit hoher Nachfrage lockt Kartenschwarzh{\"a}ndler an, welche sich Tickets ansammeln und diese dann f{\"u}r ein Mehrfaches des Ankaufpreises wiederverkaufen. Dies kann zum einen als normaler Marktmechanismus betrachtet werden, zum anderen jedoch ist es ein grosses Besorgnis von vielen K{\"u}nstlern und Organisatoren, die die Kontrolle {\"u}ber ihre Ticketpreise und deren Vertrieb verlieren. Zum Beispiel wurde Ed Sheeran's Benefizkonzert in der Albert Hall (UK) im Jahre 2017 f{\"u}r eine jugendliche Krebs Stiftung von Spekulanten missbraucht. Tickets, die urspr{\"u}nglich 75 Pfund kosteten, wurden f{\"u}r 2330 Pfund auf dem Sekund{\"a}rmarkt wiederverkauft \cite{ed-sheeran-concert-ticket-prices}. Zus{\"a}tzlich zum Ruf und dem Fairnessbedenken sind Sekund{\"a}rm{\"a}rkte auch problematisch f{\"u}r m{\"o}gliche Ticketf{\"a}lschungen. Meist kann zudem auch nicht validiert werden, ob ein Ticket gef{\"a}lscht ist oder bereits eingel{\"o}st wurde.

Die Blockchain (BC) Technologie bietet die M{\"o}glichkeit diese Probleme zu beheben. In diesem Projekt wird eine {\"o}ffentliche und dezentralisierte Plattform entworfen, implementiert und evaluiert, welche es erm{\"o}glicht, Tickets von Events zu vertreiben und deren Weiterverkauf zu regulieren. Dieses System basiert auf Smart Contracts (SC) auf der Ethereum BC, die Benutzern erlauben, die Validit{\"a}t von Tickets zu verifizieren und den sicheren Weiterverkauf dieser zu erm{\"o}glichen. Zudem wird ein Mechanismus diskutiert, der es erlaubt, den Organisator eines Events zu verifizieren und damit die Validit{\"a}t eines Events. Abschliessend wurde ein System entwickelt und implementiert, welches die effiziente Invalidierung von Tickets am Veranstaltungsort erlaubt.

\selectlanguage{english}
