\section{ID Approver}\label{section:id-approver-design}
% author: Nicolas Spielmann

The core idea of an ID approver is to allow guests to verify claims of ownership of an identity and store the linked Ethereum address with the provided identity, once the verification process was completed successfully. This is, to avoid the misuse of identities such as using it multiple times. In the following section, the concept of identity is described.

\subsection{Identity}\label{design:identity}
Since the goal of the project was to come up with a decentralized ticketing platform, a decentralized design of identity was needed. Therefore, following approach of identity and identity verification was provided.

Identity is a chain of claims and proofs. For example, to obtain a passport for a newborn child, the birth certificate of the child is required. In this example, the birth certificate is the proof. The government then uses this proof, checks it and then issues the requested passport. As it is not feasible to request the birth certificate of every event guest, it is suggested to use something, that goes through a KYC process when obtaining it. One of these products would be a phone number, since (in Switzerland) those can only be obtained when providing an identity. Therefore, an identity of a guest can be verified by checking, whether he has ownership of the phone number he claims to possess. Once this check is successful, the level of identity verification of the linked Ethereum address is set on the identity SC. This has the benefit that only integers that represent a level of proof linked to the Ethereum address need to be stored on the BC. Confidential data is only stored off chain by the ID approver.

To achieve a decentralized solution for approving identities, the designed approach allows to automate the identity approving process and allow anybody to act as an ID approver. Hence, anybody may register on the identity SC as an ID approver and also use its own idea and implementation of identity proofs.
Ultimately, this leads to the host to decide which ID approver he wants to use for his event.

\subsection{Sample Design}
%For the scope of the project, the idea was to provide an easy to run example implementation of this automated ID approver application. The sample implementation knows three different levels of identity.
Given the scope of the project, an easy way to run an example of an automated ID approver application was designed. The sample ID approver is designed to support the following three levels of identity.

\begin{itemize}
    \item The first level of identity is the verification of an email address. To verify the ownership, the application uses the native Spring Boot mail implementation.
    
    \item The second level of identity is the verification of a phone number. Twilio\footnote{\href{https://www.twilio.com/}{https://www.twilio.com/}}, an easy to use API for Spring Boot, is used to send a text message to a phone number.
    
    \item The third level of identity is the verification through a KYC-like process. The Optical character recognition (OCR) software Tesseract\footnote{\href{https://en.wikipedia.org/wiki/Tesseract_(software)}{https://en.wikipedia.org/wiki/Tesseract\_(software)}} is used to extract text from a machine readable part of a document. AWS is then used to calculate the similarity of a selfie and the picture on the passport.
\end{itemize}
