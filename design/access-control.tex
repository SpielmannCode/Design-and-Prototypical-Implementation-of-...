\section{Access Control}\label{section:access-control}



\begin{figure}[H]
    \centering
    \includegraphics[width=14cm]{design/diagrams/AcessControl.png}
    \caption{Access Control Sequence Diagram}
    \label{fig:access-controll}
\end{figure}

As the Guest wants to gain access to the event, he is confronted with the access control. As the ticket is linked to the ethereum address of the guest, he just have to proof the ownership of the given ethereum address. In order to do this, following system is proposed.

The access control terminals are registered at the back end. Every terminal gets its unique identifier. The terminal then requests an unique random sequence, which is stored with the corresponding terminal id in the back end. This random sequence is then displayed alongside the back end URL as a QR-code on the terminal. The guest reads the QR-Code and extracts the random sequence. He then proceeds to sign the sequence using his ethereum address proofing his ownership. The signature, the ethereum address and the random sequence are then sent to the back end. There, the validity of the signature is evaluated, it is checked, whether the random sequence exists and the block chain is queried, to check, whether the ethereum address actually holds a ticket. It is also checked, whether this ticket already entered the venue. When all these checks are successful, the terminal, specified by its id, is messaged to let the guest pass. At the same time, the ethereum address and the corresponding ticket are added to the database, that tracks the area the ticket owner are in (Area Control).

