\section{Surveys}

\subsection{Surveys with Event Organizers}

Part of the evaluation process was an interview and proof of concept demo with a system engineer at Gurten Festival. Their dissatisfaction with the current ticketing industry in Switzerland has driven them to building their own ticketing system. Their key motivation is to remove middlemen from the process. These middlemen take large cuts and lack of integration possibilities of their services. This system is currently in development and shares similar concepts with the proposed design. For example a ticket can be issued and resold on the aftermarket on the same platform. This allows the application to unify the ticket distribution with the aftermarket. However, the aftermarket is not limited to their platform. Tickets can also be resold for higher prices on other platforms. Their system is not able to prevent high aftermarket prices since paper tickets that are not linked to any form of identity are a requirement by the festival. This requirement is important due to the fact that less staff is required as well as the entrance queues have to be as short as possible. 

The Gurten Festival does not experience a lot of ticket fraud in terms of fake tickets. However, their presale tickets are usually sold out within a few minutes. Most of the tickets that appear on the aftermarket are sold for more than the original price.

Another insight was their opinion about personalized tickets which has changed since the Covid-19 pandemic. They expect less resistance from the public when it comes to registering the identity or phone number of event guests. Thus, personalized tickets may be more accepted in the future than their previous experience. They used personalized tickets before but the guests were unhappy since the guests had to contact the festival for changing the name on the ticket and the festival had to recruit new staff for dealing with these kind of problems. 

The system architects at Gurten Festival also thought about a blockchain powered design. However, the advantages of an open ledger came short to the more efficient nature of a client-server architecture.

The key takeaways from the interview is that there are many inefficiencies in the ticketing industry. The proposed design is radical in terms of decentralization. Trusted entities are removed as much as possible. Paper tickets are not supported with the proposed solution. One of the challenges which a ticketing platform must offer for the average event guest of today is that is does not preclude event guests which do not have the technical know-how or hardware to buy a ticket.

\subsection{Surveys with Event Guests}

To evaluate our concepts and solutions, we created a survey to target the potential guest of events. Our intent was to ask them about our identified issues and then question whether they also experienced the described issues. To achieve this, we created a survey, where we question potential customers about their thoughts about some concepts implemented in our solution. First we evaluated whether they are part of our target group by evaluating their attendance to events and interest. 
Second, we asked about the problems present in with the secondary marked and asked whether they already encountered this issue. The third section of our survey aimed to asses, whether guests would agreed to provide their personal information in order to obtain a  ticket. Then we described our approach of presale an asked if they would think of this as a fair way to distribute ticket when there are more person interested than ticket available. 

After these initial questions, we propose our solution approach and asked about their opinion and the biggest flaw in out approach.

\subsubsection{Results}
The participants of our survey all identified as potential guest, as all of them are attending at least three to five events per year. As we asked about their experience with the secondary market, we identified, that \textbf{41.7\%} already had an issue, where they could not buy a ticket due to slow internet connection or other issues such as scalpers buying all the tickets. They also stated, that when they searched for ticket on secondary markets, they only saw listings for a higher price than initially offered on the primary market.

As we asked about personalized tickets, we wanted to know, whether they are fine with having a personalized ticket. All of them were fine with having this, as long as the entity was trustful. The entities perceived most trustful were the government entities such as the canton or the municipality. 

Our idea of presale was the most controversial concept on during the whole survey. Only a third of the participants preferred a fair presale. This is mostly due to the uncertainty the presale brings. Most participants liked it more to just be ready at any given time, since they experienced to always get a ticket. 

Our Concept of a fair distribution of profit with the gui-providers, affiliates and the Identity Approvers was received very positively. They thought our model was more fair. 

However, the biggest problem we identified was our core technologie used. A lot of participants never heard of cryptocurrencies. They identified this as the biggest issue, as cryptocurrencies are just not established enough (yet). Contratry to that, we often heard, that depending on the artist, people tend to overcome any obstacles in order to obtain a ticket.




