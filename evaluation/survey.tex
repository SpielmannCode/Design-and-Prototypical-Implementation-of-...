\section{Surveys}

\subsection{Surveys with Event Organizers}

Part of the evaluation process was an interview and proof of concept demo with a system engineer at Gurten Festival. Their dissatisfaction with the current situation of the ticketing industry in Switzerland has driven them to building their own ticketing system. Their key motivation is to remove middlemen from the process. These middlemen take large cuts and lack of integration possibilities of their services. This system is currently in development and shares similar concepts with the proposed design of this project. For example a ticket can be issued and resold on the aftermarket on the same platform. This allows the application to unify the ticket distribution with the aftermarket. However, the aftermarket is not limited to their platform. Tickets can also be resold for higher prices on other platforms. Their system is not able to prevent high aftermarket prices since paper tickets that are not linked to any form of identity are a requirement by the festival. This requirement is important due to the fact that less staff is required as well as the entrance queues have to be as short as possible. 

The Gurten Festival does not experience a lot of ticket fraud in terms of fake tickets. However, their presale tickets are usually sold out within a few minutes. Most of the tickets that appear on the aftermarket are sold for more than the original price.

Another insight was their opinion about personalized tickets which has changed since the Covid-19 pandemic. They expect less resistance from the public when it comes to registering the identity or phone number of event guests. Thus, personalized tickets may be more accepted in the future than their previous experience. They used personalized tickets before but the guests were unhappy since the guests had to contact the festival for changing the name on the ticket and the festival had to recruit new staff for dealing with these kind of problems. 

The system architects at Gurten Festival also thought about a BC powered design. However, the advantages of an open ledger came short to the more efficient nature of a client-server architecture.

The key takeaways from the interview is that there are many inefficiencies in the ticketing industry. The proposed design is radical in terms of decentralization. Trusted entities are removed as much as possible. Paper tickets are not supported with the proposed solution. One of the challenges which a ticketing platform must offer for the average event guest of today is that is does not preclude event guests which do not have the technical know-how or hardware to buy a ticket.

\subsection{Surveys with Event Guests}

To evaluate the concepts and solutions produced in this project, a survey was conducted to target potential guests of events. The intent was to ask them, whether they experienced the described issues and what they like or dislike about the solution approach examined and implemented in this project.

First, it was evaluated whether they are part of the general target group. Second, they were asked about the problems present in the secondary marked and whether they already encountered such an issue. The third section of our survey aimed to assess whether guests would agree to provide their personal information in order to obtain a ticket. Further, the proposed approach of a presale was described and it was asked if they would think of this as a fair way to distribute ticket in cases where the demand is higher than the supply.

After these initial questions, the proposed solution approach was presented and the participants were asked about their opinion and the biggest flaws in the proposed approach.

\subsubsection{Results}
The 13 participants of the survey all identified as potential guests as all of them are attending at least three to five events per year. As they were asked about their experience with the secondary market, the presence of this issue was confirmed. 41.7\% of the participants have already experienced an issue, where they could not buy a ticket due to a slow internet connection or other issues such as scalpers buying many tickets and then offer them on a secondary market. They also stated, that when they searched for tickets on secondary markets, they only found listings for a higher price than initially offered on the primary market.

All of the participants would support personalized tickets, as long as the entity that checks the identity was trustful. The entities perceived most trustful were government entities such as a canton or a municipality.

The proposed idea of a presale was the most controversial concept in the conducted survey. Only a third of the participants preferred a fair presale. They argued, that this was mostly due to the uncertainty a presale brings and they would have to wait until they know whether they actually get a ticket or not. Most of them liked it more to just be ready at any given time, since they experienced to always get a ticket. 

The proposed concept of a fair distribution of profit with the GUI-providers, affiliates and the ID approvers was received very positively. The proposed model was preceived as being fair.

Finally, the biggest problem identified was the core technology used. A lot of participants have never heard of cryptocurrencies or at least do not know much about it. This was identified as the biggest issue, as cryptocurrencies are just not established enough. Contrary to that, it was often stated, that depending on the artist, people tend to overcome any obstacles in order to obtain a ticket.
