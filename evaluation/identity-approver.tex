\section{ID approver}\label{sec:identity-approver}
The main concern of the design and implementation of the ID approver was the cost an ID approver has to verify a potential guest. This section describes the issues faced and evaluates the solutions implemented.

\subsection{Cost Analysis}
As the ID approver stores the proofs to the BC, transaction costs are payed by the ID approver. The ID approver also has to pay for the third party services used to verify a guest's identity. Therefore, the price for one validation of each tier was calculated.

As described in Section \ref{sc:econom}, the ID approver gets compensated, whenever a guest, that has been verified by him, buys a ticket. All the affiliates together are compensated with 10\% of the ticket price. Therefore, the amount received by the ID approver decreases when the number of affiliates for this ticket sale increases. For this evaluation, it is assumed that there are only two affiliates beside the ID approver, giving him 3.3\% of the ticket price as reward. Three affiliates are chosen, since this is the most likely scenario, where one affiliate is the GUI provider and the other affiliate beside him is a promoter that promotes the event.

\subsubsection{Email}
The verification of the guest's email address is the least complex interaction. Given, that a ID approver already has an existing email server, it is assumed that the cost for an additional email address is negligible. Therefore, only the gas cost of the BC transaction is factored into the calculation. The function to approve a guest consumes a total amount of \textbf{28'845 Gas}. 

Based on the ETH price of the evaluation (see Section \ref{sec:smart_cont:eval}), this results in a cost of \textbf{0.00098 ETH} which is equivalent to approximately \textbf{\$0.37}. This means, that the ID approver already paid for his expenses, once the guest bought one ticket that cost more than \textbf{\$12}. 

\subsubsection{Phone}
The Ethereum transaction cost is equal regardless of what level of identity is approved. Hence, this cost stays at \textbf{\$0.37}. However, on top of this, the cost of sending out text messages has to be considered. Twilio charges \textbf{\$0.069} per SMS sent and \textbf{\$8} per phone number per month \cite{twillio-Cost}. With a very pessimistic calculation of only 10 verifications per month, this results in a total cost of \textbf{\$0.869} per text message.

This results in a total cost of \textbf{\$1.239}, which means that the ID approver has already paid for his expenses, once the guest bought one ticket that costs more than \textbf{\$41}.

\subsubsection{KYC}
The difference of a KYC verification compared to the other tiers is, that AWS rekognition is used to calculate similarities of faces in pictures. The cost of this service reduces with the amount of pictures analyzed. When considering a pessimistic evaluation, the price of the lowest tier and a european AWS server is used to calculate the cost. This results in a cost of \textbf{\$0.0012} per image\cite{AWS-Cost}.

This results in a total cost of \textbf{\$0.3712}, which means, that the ID approver has already paid for his expenses, once the guest bought one ticket that cost more than \textbf{\$12}.

\subsubsection{Conclusion}
The average ticket price in the year 2017 was \textbf{79.47 CHF} which is around \textbf{\$88} \cite{Ticket-Price}. This means, that the ID approver has a very high chance to break even or make profit as soon as a guest buys a ticket. Therefore, the suggested model seems to be very profitable for an ID approver, since they only need to verify the guest once, but they will get a share of the ticket price anytime an identity verification of him as an ID approver is required.
