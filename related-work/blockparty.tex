% author: Michael Bucher

\section{Blockparty}

Blockparty \cite{blockparty-whitepaper} is a blockchain based event ticketing platform built on the Ethereum blockchain. They want to solve current problems that are arising in the ticketing industry - mainly fraud, and \textit{bulk-buying} by bots. Their goal is to provide a seemless experience for event-goers and event hosts.

\paragraph{Storage} Blockparty uses IPFS to store metadata from each user, event and ticket. Each ticket itself is a smart contract that handles its transfer and allocation to a user. Whenever a ticket is sold or transferred, the contract is invoked accordingly and the new data is stored on the blockchain.

\paragraph{Identity} To register and interact with Blockparty, a user first has to provide either a fingerprint or facial recognition. Blockparty calls either of those \textit{digital fingerprint}. This is required for buying a ticket, since Blockparty links your data to each ticket you buy. To ensure anonymity the digital fingerprint is encrypted and then hashed before linking it to the ticket. Using such biometric data, Blockparty expects to eliminate counterfeit and fraud to a great amount, since \textit{e.g.} bots could be avoided because of lack of identity provisioning.

\paragraph{Ticket Transfer} There are many justifiable reasons, why a ticket holder might want to transfer or resell his owned ticket. For this matter, Blockparty offers a direct peer-to-peer transfer from one Blockparty account to another. Upon registering on Blockparty, a user has to provide a mobile phone number. This is used as your Blockparty ID. A user can now send an owned ticket to a mobile phone number that is linked to another Blockparty account. Upon accepting a received ticket, the ticket is unlinked from the sender's digital fingerprint and linked to the receiver's digital fingerprint.

\paragraph{UX} Blockparty currently provides an app for event-goers and an app for event hosts. Former is used for buying and holding tickets and showing relevant information about events. The latter is used by the host and its staff to scan and validate tickets that are presented on the former app by the guest.

\paragraph{Governance} Blockparty provides the centralized service of account recovery in case of loss of a users mobile phone.

\paragraph{BOXX Token} The BOXX token is used as a transaction utility. There is a fixed volume of 90 million BOXX tokens. In an introductory phase this token is used as reward for interacting with Blockparty, such as hosting an event, buying a ticket or promoting an event. Later on the token should be used in the exchange mechanism within Blockparty.

\paragraph{Conclusion} Blockparty provides some interesting solutions to current problems in the event industy. However, it lacks of transparency and actual implementation of their solutions. To register and buy a ticket one does not actually have to provide any fingerprint or facial recognition. If a user wants to buy a ticket, it is only possible solely with credit card and there is no mentioning of the BOXX token or any Ethereum wallet. Further, a ticket can be transferred to any mobile phone number without any transaction details given, which ultimately means that the black market is entirely ignored by their protocol.
