% author: Claudio Brasser

\section{Aventus}

\subsection{Problems}
\paragraph{Oversight}

The main problem here is defined as a lack of oversight over digital assets from the issuer of these assets/. Oversight isn this context is defined as the ability for an issuer see exactly what is happening to their asset at any given moment in order to derive increased value from their digital assets.

\paragraph{Control}

Control allows an issuer to protect the value of their digital assets. In order to achieve total control, the issuer needs control over how their assets are created, managed, and sold.

A lack of control can lead to problems such as distribution of fake copies, price inflation in reselling markets, distribution by unauthorised vendors.

\subsection{The Aventus Protocol}

The Aventus protocol is built on top of an existing public Blockchain network (Ethereum) and handles transactions off-chain.
-> The reason it does this is to overcome the problem of scalability on Ethereum, while still benefiting from its security and independence.

Transactions are processed and verified more efficiently by Aventus nodes and then batched together into Aventus Blocks and sent to the Ethereum Blockchain, were they are made public.

The Aventus Protocol uses zero-knowledge proofs to adress the problem of privacy on the Ethereum Blockchain and prove ownership without leaking extra information. (Sigma Protocol)
-> These proofs allow users to add confidential information to transactions without revealing this information to the public, and while still allowing for the transaction to be publicly validated


\paragraph{Aventus Nodes}



