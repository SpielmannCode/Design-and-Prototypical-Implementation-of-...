\section{Other Projects}

There exist many other projects that provide an approach on how to connect the ticketing industry with blockchain technology.

\paragraph{Bitticket}
The Bitticket service\cite{bitticket} is prominently mentioned on the website of the ticketing company \textit{Citizen Ticket}, mainly active within the United Kingdom. The company is already hosting a plethora of different events, although at the time being most events taking place online. When buying a ticket on their website (as is through the mobile application), the user is not informed about anything related to the blockchain and payment is purely done in British pounds. Citizen Ticket did neither release a white paper nor is there any technical information available on their website, concerning how they incorporate blockchain technology into their service.

\paragraph{SecuTix}
SecuTix\cite{SECUTIX} is an established Swiss-based ticket provider that provides a cloud-based solution to issue tickets combined with customer-relationship-management. They are an established player in the ticket industry and provide and issue tickets for different events such as ice hockey namely the Lausanne Hockey Club or cultural events such as the Basel Theater. They released a white paper that suggest that using Block chain technology will address a lot of issues with the current ticketing process. Their main goal was to address ticket fraud and the secondary ticket market. However, beside the white paper, there is no sign of the usage of block chain in their ticket process.

\paragraph{Espass} 
Espass\cite{espass} is less of a company that focus on solving the ticketing problem on the block chain but rather how to have a digital representation of passes for events, boarding or just coupons. However, the GitHub repository has not been touched in a long while and seems to be only maintained by a single developer. No in use applications have been found. The Website is rather used to describe the way the want to implement digital passes.  

\paragraph{Ticket chain}
Ticket chain\cite{TicketChain} was initial  focused on providing ticketing via the block chain. However in December 2017 they officially declared to move away from the block chain technology and are now focusing on providing the venues and artist a way to control their primary and secondary ticket market. There is not a lot of technical information about their block chain implementation present. 

\paragraph{True Tickets}
True Tickets\cite{truetickets} was founded in 2017 and has been prominently mentioned on IBM blog entries \cite{truetickets-ibm}, as it is built upon the IBM Blockchain. They advertise their platform as a solution to prominent problems in the ticketing industry, such as counterfeiting and price inflation on the reselling market. True Tickets does not seem to have hosted any events at the time being, however they have been cited to launch a pilot program in corporation with a major ticketing organization in 2020 \cite{truetickets-pilot}.
Neither on their website nor in related blog posts has there been hints on publicly available development insights containing technical details. The official code repository on GitHub\cite{github} is private as well.



\paragraph{Event Token}
Event Token is also a Swiss-based startup that has launched a mobile application for iOS and Android. According to their website, an event with more than 7000 guests was hosted on the platform. However, no implementation details or architectural designs are published. Furthermore, when using the mobile application, the user is not confronted with blockchain wallets. 


\paragraph{UPGRADED} UPGRADED is part of the well-known companies Ticketmaster and Live Nations. They provide a blockchain based solution for ticket management and tr in one app to avoid fraudulent behavior. However, it is not communicated, how exactly the blockchain technology is used.

TODO paragraph/single sentence:
Nici: Ticket chain, Espass

The following projects pursued a similar goal. They are either no longer maintained or there is not enough information available to include in the analysis. 
\begin{itemize}
    \item Crypto Tickets
    \item Hello Sugoi
    \item EventX
    \item LAVA
\end{itemize}